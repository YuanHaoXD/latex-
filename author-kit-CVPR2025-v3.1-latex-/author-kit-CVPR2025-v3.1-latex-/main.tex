\documentclass[10pt,twocolumn,letterpaper]{article}

\usepackage{cvpr}      % To produce the REVIEW version
%\usepackage 用来加载 cvpr 宏包。
% \usepackage[review]{cvpr} 表示将生成用于评审(review)的版本,会带有行号。
% \usepackage{cvpr} 是最终提交时的版本,不带行号。
% \usepackage[pagenumbers]{cvpr} 添加页码,一般用于预印版(如 arXiv)。
%
% --- inline annotations
%
\newcommand{\red}[1]{{\color{red}#1}}
\newcommand{\todo}[1]{{\color{red}#1}}
\newcommand{\TODO}[1]{\textbf{\color{red}[TODO: #1]}}
% --- disable by uncommenting  
% \renewcommand{\TODO}[1]{}
% \renewcommand{\todo}[1]{#1}



\definecolor{cvprblue}{rgb}{0.21,0.49,0.74}
\usepackage[pagebackref,breaklinks,colorlinks,allcolors=cvprblue]{hyperref}
%----------------------设置中文字体--------------------------
\usepackage{fontspec}
\usepackage{xunicode}
\usepackage{xeCJK}

% 设置英文字体
\setmainfont{Times New Roman}

% 设置中文字体
\setCJKmainfont[AutoFakeBold]{SimSun}    % 宋体
\setCJKsansfont{SimHei}   % 黑体
\setCJKmonofont{FangSong} % 仿宋

%--------------------------------------------------------------------------
\usepackage{setspace}
% 单倍行距
\onehalfspacing
\usepackage {indentfirst}

%%%%%%%%% PAPER ID  - PLEASE UPDATE
\def\paperID{1} % *** Enter the Paper ID here
\def\confName{CVPR}
\def\confYear{2025}
%三个宏定义,用于在底下进行替换
%%%%%%%%% TITLE - PLEASE UPDATE

% \title{\LaTeX\ Author Guidelines for \confName~Proceedings}
\title{学习记录}

%输入标题
%%%%%%%%% AUTHORS - PLEASE UPDATE
\author{王远皓
% Institution1\\
% Institution1 address\\
% {\tt\small firstauthor@i1.org}
% \and
% Second Author\\
% Institution2\\
% First line of institution2 address\\
% {\tt\small secondauthor@i2.org}
}

\begin{document}
\maketitle
% \begin{abstract}
The ABSTRACT is to be in fully justified italicized text, at the top of the left-hand column, below the author and affiliation information.
Use the word ``Abstract'' as the title, in 12-point Times, boldface type, centered relative to the column, initially capitalized.
The abstract is to be in 10-point, single-spaced type.
Leave two blank lines after the Abstract, then begin the main text.
Look at previous \confName  abstracts to get a feel for style and length.
\end{abstract}    
% \section{Introduction}
\label{sec:intro}

Please follow the steps outlined below when submitting your manuscript to the IEEE Computer Society Press.
This style guide now has several important modifications (for example, you are no longer warned against the use of sticky tape to attach your artwork to the paper), so all authors should read this new version.

%-------------------------------------------------------------------------
\subsection{Language}

All manuscripts must be in English.

\subsection{Dual submission}

Please refer to the author guidelines on the \confName\ \confYear\ web page for a
discussion of the policy on dual submissions.

\subsection{Paper length}
Papers, excluding the references section, must be no longer than eight pages in length.
The references section will not be included in the page count, and there is no limit on the length of the references section.
For example, a paper of eight pages with two pages of references would have a total length of 10 pages.
{\bf There will be no extra page charges for \confName\ \confYear.}

Overlength papers will simply not be reviewed.
This includes papers where the margins and formatting are deemed to have been significantly altered from those laid down by this style guide.
Note that this \LaTeX\ guide already sets figure captions and references in a smaller font.
The reason such papers will not be reviewed is that there is no provision for supervised revisions of manuscripts.
The reviewing process cannot determine the suitability of the paper for presentation in eight pages if it is reviewed in eleven.

%-------------------------------------------------------------------------
\subsection{The ruler}
The \LaTeX\ style defines a printed ruler which should be present in the version submitted for review.
The ruler is provided in order that reviewers may comment on particular lines in the paper without circumlocution.
If you are preparing a document using a non-\LaTeX\ document preparation system, please arrange for an equivalent ruler to appear on the final output pages.
The presence or absence of the ruler should not change the appearance of any other content on the page.
The camera-ready copy should not contain a ruler.
(\LaTeX\ users may use options of \texttt{cvpr.sty} to switch between different versions.)

Reviewers:
note that the ruler measurements do not align well with lines in the paper --- this turns out to be very difficult to do well when the paper contains many figures and equations, and, when done, looks ugly.
Just use fractional references (\eg, this line is $087.5$), although in most cases one would expect that the approximate location will be adequate.


\subsection{Paper ID}
Make sure that the Paper ID from the submission system is visible in the version submitted for review (replacing the ``*****'' you see in this document).
If you are using the \LaTeX\ template, \textbf{make sure to update paper ID in the appropriate place in the tex file}.


\subsection{Mathematics}

Please number all of your sections and displayed equations as in these examples:

\begin{equation}
  E = m\cdot c^2
  \label{eq:important}
\end{equation}

and
\begin{equation}
  v = a\cdot t.
  \label{eq:also-important}
\end{equation}
It is important for readers to be able to refer to any particular equation.
Just because you did not refer to it in the text does not mean some future reader might not need to refer to it.
It is cumbersome to have to use circumlocutions like ``the equation second from the top of page 3 column 1''.
(Note that the ruler will not be present in the final copy, so is not an alternative to equation numbers).
All authors will benefit from reading Mermin's description of how to write mathematics:
\url{http://www.pamitc.org/documents/mermin.pdf}.

\subsection{Blind review}

Many authors misunderstand the concept of anonymizing for blind review.
Blind review does not mean that one must remove citations to one's own work---in fact it is often impossible to review a paper unless the previous citations are known and available.

Blind review means that you do not use the words ``my'' or ``our'' when citing previous work.
That is all.
(But see below for tech reports.)

Saying ``this builds on the work of Lucy Smith [1]'' does not say that you are Lucy Smith;
it says that you are building on her work.
If you are Smith and Jones, do not say ``as we show in [7]'', say ``as Smith and Jones show in [7]'' and at the end of the paper, include reference 7 as you would any other cited work.

An example of a bad paper just asking to be rejected:
\begin{quote}
\begin{center}
    An analysis of the frobnicatable foo filter.
\end{center}

   In this paper we present a performance analysis of our previous paper [1], and show it to be inferior to all previously known methods.
   Why the previous paper was accepted without this analysis is beyond me.

   [1] Removed for blind review
\end{quote}


An example of an acceptable paper:
\begin{quote}
\begin{center}
     An analysis of the frobnicatable foo filter.
\end{center}

   In this paper we present a performance analysis of the  paper of Smith \etal [1], and show it to be inferior to all previously known methods.
   Why the previous paper was accepted without this analysis is beyond me.

   [1] Smith, L and Jones, C. ``The frobnicatable foo filter, a fundamental contribution to human knowledge''. Nature 381(12), 1-213.
\end{quote}

If you are making a submission to another conference at the same time, which covers similar or overlapping material, you may need to refer to that submission in order to explain the differences, just as you would if you had previously published related work.
In such cases, include the anonymized parallel submission~\cite{Authors14} as supplemental material and cite it as
\begin{quote}
[1] Authors. ``The frobnicatable foo filter'', F\&G 2014 Submission ID 324, Supplied as supplemental material {\tt fg324.pdf}.
\end{quote}

Finally, you may feel you need to tell the reader that more details can be found elsewhere, and refer them to a technical report.
For conference submissions, the paper must stand on its own, and not {\em require} the reviewer to go to a tech report for further details.
Thus, you may say in the body of the paper ``further details may be found in~\cite{Authors14b}''.
Then submit the tech report as supplemental material.
Again, you may not assume the reviewers will read this material.

Sometimes your paper is about a problem which you tested using a tool that is widely known to be restricted to a single institution.
For example, let's say it's 1969, you have solved a key problem on the Apollo lander, and you believe that the 1970 audience would like to hear about your
solution.
The work is a development of your celebrated 1968 paper entitled ``Zero-g frobnication: How being the only people in the world with access to the Apollo lander source code makes us a wow at parties'', by Zeus \etal.

You can handle this paper like any other.
Do not write ``We show how to improve our previous work [Anonymous, 1968].
This time we tested the algorithm on a lunar lander [name of lander removed for blind review]''.
That would be silly, and would immediately identify the authors.
Instead write the following:
\begin{quotation}
\noindent
   We describe a system for zero-g frobnication.
   This system is new because it handles the following cases:
   A, B.  Previous systems [Zeus et al. 1968] did not  handle case B properly.
   Ours handles it by including a foo term in the bar integral.

   ...

   The proposed system was integrated with the Apollo lunar lander, and went all the way to the moon, don't you know.
   It displayed the following behaviours, which show how well we solved cases A and B: ...
\end{quotation}
As you can see, the above text follows standard scientific convention, reads better than the first version, and does not explicitly name you as the authors.
A reviewer might think it likely that the new paper was written by Zeus \etal, but cannot make any decision based on that guess.
He or she would have to be sure that no other authors could have been contracted to solve problem B.
\medskip

\noindent
FAQ\medskip\\
{\bf Q:} Are acknowledgements OK?\\
{\bf A:} No.  Leave them for the final copy.\medskip\\
{\bf Q:} How do I cite my results reported in open challenges?
{\bf A:} To conform with the double-blind review policy, you can report results of other challenge participants together with your results in your paper.
For your results, however, you should not identify yourself and should not mention your participation in the challenge.
Instead present your results referring to the method proposed in your paper and draw conclusions based on the experimental comparison to other results.\medskip\\

\begin{figure}[t]
  \centering
  \fbox{\rule{0pt}{2in} \rule{0.9\linewidth}{0pt}}
   %\includegraphics[width=0.8\linewidth]{egfigure.eps}

   \caption{Example of caption.
   It is set in Roman so that mathematics (always set in Roman: $B \sin A = A \sin B$) may be included without an ugly clash.}
   \label{fig:onecol}
\end{figure}

\subsection{Miscellaneous}

\noindent
Compare the following:\\
\begin{tabular}{ll}
 \verb'$conf_a$' &  $conf_a$ \\
 \verb'$\mathit{conf}_a$' & $\mathit{conf}_a$
\end{tabular}\\
See The \TeX book, p165.

The space after \eg, meaning ``for example'', should not be a sentence-ending space.
So \eg is correct, {\em e.g.} is not.
The provided \verb'\eg' macro takes care of this.

When citing a multi-author paper, you may save space by using ``et alia'', shortened to ``\etal'' (not ``{\em et.\ al.}'' as ``{\em et}'' is a complete word).
If you use the \verb'\etal' macro provided, then you need not worry about double periods when used at the end of a sentence as in Alpher \etal.
However, use it only when there are three or more authors.
Thus, the following is correct:
   ``Frobnication has been trendy lately.
   It was introduced by Alpher~\cite{Alpher02}, and subsequently developed by
   Alpher and Fotheringham-Smythe~\cite{Alpher03}, and Alpher \etal~\cite{Alpher04}.''

This is incorrect: ``... subsequently developed by Alpher \etal~\cite{Alpher03} ...'' because reference~\cite{Alpher03} has just two authors.

\begin{figure*}
  \centering
  \begin{subfigure}{0.68\linewidth}
    \fbox{\rule{0pt}{2in} \rule{.9\linewidth}{0pt}}
    \caption{An example of a subfigure.}
    \label{fig:short-a}
  \end{subfigure}
  \hfill
  \begin{subfigure}{0.28\linewidth}
    \fbox{\rule{0pt}{2in} \rule{.9\linewidth}{0pt}}
    \caption{Another example of a subfigure.}
    \label{fig:short-b}
  \end{subfigure}
  \caption{Example of a short caption, which should be centered.}
  \label{fig:short}
\end{figure*}

% \section{Formatting your paper}
\label{sec:formatting}

All text must be in a two-column format.
The total allowable size of the text area is $6\frac78$ inches (17.46 cm) wide by $8\frac78$ inches (22.54 cm) high.
Columns are to be $3\frac14$ inches (8.25 cm) wide, with a $\frac{5}{16}$ inch (0.8 cm) space between them.
The main title (on the first page) should begin 1 inch (2.54 cm) from the top edge of the page.
The second and following pages should begin 1 inch (2.54 cm) from the top edge.
On all pages, the bottom margin should be $1\frac{1}{8}$ inches (2.86 cm) from the bottom edge of the page for $8.5 \times 11$-inch paper;
for A4 paper, approximately $1\frac{5}{8}$ inches (4.13 cm) from the bottom edge of the
page.

%-------------------------------------------------------------------------
\subsection{Margins and page numbering}

All printed material, including text, illustrations, and charts, must be kept
within a print area $6\frac{7}{8}$ inches (17.46 cm) wide by $8\frac{7}{8}$ inches (22.54 cm)
high.
%
Page numbers should be in the footer, centered and $\frac{3}{4}$ inches from the bottom of the page.
The review version should have page numbers, yet the final version submitted as camera ready should not show any page numbers.
The \LaTeX\ template takes care of this when used properly.



%-------------------------------------------------------------------------
\subsection{Type style and fonts}

Wherever Times is specified, Times Roman may also be used.
If neither is available on your word processor, please use the font closest in
appearance to Times to which you have access.

MAIN TITLE.
Center the title $1\frac{3}{8}$ inches (3.49 cm) from the top edge of the first page.
The title should be in Times 14-point, boldface type.
Capitalize the first letter of nouns, pronouns, verbs, adjectives, and adverbs;
do not capitalize articles, coordinate conjunctions, or prepositions (unless the title begins with such a word).
Leave two blank lines after the title.

AUTHOR NAME(s) and AFFILIATION(s) are to be centered beneath the title
and printed in Times 12-point, non-boldface type.
This information is to be followed by two blank lines.

The ABSTRACT and MAIN TEXT are to be in a two-column format.

MAIN TEXT.
Type main text in 10-point Times, single-spaced.
Do NOT use double-spacing.
All paragraphs should be indented 1 pica (approx.~$\frac{1}{6}$ inch or 0.422 cm).
Make sure your text is fully justified---that is, flush left and flush right.
Please do not place any additional blank lines between paragraphs.

Figure and table captions should be 9-point Roman type as in \cref{fig:onecol,fig:short}.
Short captions should be centred.

\noindent Callouts should be 9-point Helvetica, non-boldface type.
Initially capitalize only the first word of section titles and first-, second-, and third-order headings.

FIRST-ORDER HEADINGS.
(For example, {\large \bf 1. Introduction}) should be Times 12-point boldface, initially capitalized, flush left, with one blank line before, and one blank line after.

SECOND-ORDER HEADINGS.
(For example, { \bf 1.1. Database elements}) should be Times 11-point boldface, initially capitalized, flush left, with one blank line before, and one after.
If you require a third-order heading (we discourage it), use 10-point Times, boldface, initially capitalized, flush left, preceded by one blank line, followed by a period and your text on the same line.

%-------------------------------------------------------------------------
\subsection{Footnotes}

Please use footnotes\footnote{This is what a footnote looks like.
It often distracts the reader from the main flow of the argument.} sparingly.
Indeed, try to avoid footnotes altogether and include necessary peripheral observations in the text (within parentheses, if you prefer, as in this sentence).
If you wish to use a footnote, place it at the bottom of the column on the page on which it is referenced.
Use Times 8-point type, single-spaced.


%-------------------------------------------------------------------------
\subsection{Cross-references}

For the benefit of author(s) and readers, please use the
{\small\begin{verbatim}
  \cref{...}
\end{verbatim}}  command for cross-referencing to figures, tables, equations, or sections.
This will automatically insert the appropriate label alongside the cross-reference as in this example:
\begin{quotation}
  To see how our method outperforms previous work, please see \cref{fig:onecol} and \cref{tab:example}.
  It is also possible to refer to multiple targets as once, \eg~to \cref{fig:onecol,fig:short-a}.
  You may also return to \cref{sec:formatting} or look at \cref{eq:also-important}.
\end{quotation}
If you do not wish to abbreviate the label, for example at the beginning of the sentence, you can use the
{\small\begin{verbatim}
  \Cref{...}
\end{verbatim}}
command. Here is an example:
\begin{quotation}
  \Cref{fig:onecol} is also quite important.
\end{quotation}

%-------------------------------------------------------------------------
\subsection{References}

List and number all bibliographical references in 9-point Times, single-spaced, at the end of your paper.
When referenced in the text, enclose the citation number in square brackets, for
example~\cite{Authors14}.
Where appropriate, include page numbers and the name(s) of editors of referenced books.
When you cite multiple papers at once, please make sure that you cite them in numerical order like this \cite{Alpher02,Alpher03,Alpher05,Authors14b,Authors14}.
If you use the template as advised, this will be taken care of automatically.

\begin{table}
  \centering
  \begin{tabular}{@{}lc@{}}
    \toprule
    Method & Frobnability \\
    \midrule
    Theirs & Frumpy \\
    Yours & Frobbly \\
    Ours & Makes one's heart Frob\\
    \bottomrule
  \end{tabular}
  \caption{Results.   Ours is better.}
  \label{tab:example}
\end{table}

%-------------------------------------------------------------------------
\subsection{Illustrations, graphs, and photographs}

All graphics should be centered.
In \LaTeX, avoid using the \texttt{center} environment for this purpose, as this adds potentially unwanted whitespace.
Instead use
{\small\begin{verbatim}
  \centering
\end{verbatim}}
at the beginning of your figure.
Please ensure that any point you wish to make is resolvable in a printed copy of the paper.
Resize fonts in figures to match the font in the body text, and choose line widths that render effectively in print.
Readers (and reviewers), even of an electronic copy, may choose to print your paper in order to read it.
You cannot insist that they do otherwise, and therefore must not assume that they can zoom in to see tiny details on a graphic.

When placing figures in \LaTeX, it's almost always best to use \verb+\includegraphics+, and to specify the figure width as a multiple of the line width as in the example below
{\small\begin{verbatim}
   \usepackage{graphicx} ...
   \includegraphics[width=0.8\linewidth]
                   {myfile.pdf}
\end{verbatim}
}


%-------------------------------------------------------------------------
\subsection{Color}

Please refer to the author guidelines on the \confName\ \confYear\ web page for a discussion of the use of color in your document.

If you use color in your plots, please keep in mind that a significant subset of reviewers and readers may have a color vision deficiency; red-green blindness is the most frequent kind.
Hence avoid relying only on color as the discriminative feature in plots (such as red \vs green lines), but add a second discriminative feature to ease disambiguation.
% \section{Final copy}

You must include your signed IEEE copyright release form when you submit your finished paper.
We MUST have this form before your paper can be published in the proceedings.

Please direct any questions to the production editor in charge of these proceedings at the IEEE Computer Society Press:
\url{https://www.computer.org/about/contact}.

\begin{abstract}
本周学习内容:熟悉\LaTeX\ 的使用,制作了可供往后学习记录的模板,能提高些后面学习记录的效率
;学习并了解一些最短路径算法,包括深度优先搜索算法\textbf{DFS}、广度优先搜索算法\textbf{BFS}、
贪心算法、A*算法;最后是在课上学习了用于简化矩阵方程求解的LU拆解。
\end{abstract}    
\section{LaTeX个人使用手册(CVPR格式部分解析)}
\subsection{main.tex部分功能解析}
%分点描述:
\begin{itemize}
    \item  $\backslash$documentclas[10pt,twocolumn,letterpaper]\{article\}\\
    documentclass 定义文档类型。参数[10pt,twocolumn,letterpaper] 表示字体大小为 10pt,双栏排版,纸张大小为 letter(美国标准大小)。
    \item $\backslash$usepackage 用来加载 cvpr 宏包。$\backslash$usepackage[review]\{cvpr\} 表示将生成用于评审(review)的版本,会带有行号。\\
    $\backslash$usepackage\{cvpr\} 是最终提交时的版本,不带行号。\\
    $\backslash$usepackage[pagenumbers]\{cvpr\} 添加页码,一般用于预印版(如 arXiv)。
    \item  $\backslash$usepackage{setspace}\\
    单倍行距\\
    $\backslash$singlespacing\\
    1.5倍行距\\
    $\backslash$onehalfspacing\\
    双倍行距\\
    $\backslash$doublespacing
    \item 中文设置 \\
    $\backslash$usepackage{fontspec}\\
    $\backslash$usepackage{xunicode}\\
    $\backslash$usepackage{xeCJK}
    \\设置英文字体\\
    $\backslash$setmainfont{Times New Roman}
    \\ 设置中文字体\\
    $\backslash$setCJKmainfont[AutoFakeBold]{SimSun} 常规字体:宋体 \\  
    $\backslash$setCJKsansfont{SimHei} 加粗字体:黑体\\   
    $\backslash$setCJKmonofont{FangSong}等宽字体:仿宋
\end{itemize}
\subsection{一些有助于排版的小功能(持续更新)}
\begin{itemize}
    \item $\backslash$noindet取消段落前的缩进
    \item$\backslash$smallskip:插入一个小的空白,用于需要更小间距的情况。
 \item$\backslash$bigskip:插入一个较大的空白,用于需要较大间距的情况。
 \item$\backslash$medskip 是一个插入适中高度空白的命令,常用于分隔文本段落或元素,增强文档的可读性和排版美感。
 \item$\backslash$$\backslash$换行
 \item$\backslash$textbf{}加粗表示
 \item$\backslash$and 表示并列
 \item$\backslash$ref 表示引用
 \item 正在使用的分点表示:\\$\backslash$begin\{itemize\}\\$\backslash$item 分点内容\\$\backslash$end\{itemize\}
\end{itemize}
\section{最短路径算法}
\subsection{\textbf{搜索}}
搜索问题涉及一个智能体(\textbf{agent}),该智能体会接收初始状态和目标状态,
并返回从初始状态到目标状态的解决方案。例如,导航应用程序就是一个典型的搜索过程,其中智能体(程序的思考部分)接收当前的位置和目标目的地作为输入,并基于搜索算法返回一条推荐路径。
然而,搜索问题的形式还有很多其他类型,比如拼图或迷宫等。

搜索过程主要包括以下几个步骤:
\begin{itemize}
    \item \textbf{智能体(Agent)}:\\一个感知其环境并对环境做出反应的\textbf{实体}。例如,在导航应用程序中,
    智能体可以看作是代表汽车的一个模型,负责决定采取哪些行动以到达目的地。\smallskip
    \item \textbf{状态(State)}:\\智能体在环境中的一种\textbf{配置}。例如,
    在15拼图游戏中,状态是指所有数字在棋盘上排列的某种方式。\smallskip
    \item \textbf{行动(Action)}:\\在某个状态下可以做出的选择。更准确地说,行动可以被定义为一个\textbf{函数}。当接收状态作为输入时,返回可以在该状态下执行的\textbf{行动集合}。
    例如,在15拼图游戏中,给定状态下的行动是指在当前配置中可以滑动的方块(
    如果空白方块在中间,则可以滑动4个方块;如果在边缘,则是3个;如果在角落,则是2个)。\smallskip
    \item \textbf{转换模型(Transition Model)}:\\描述执行任何适用行动后,所得到的新状态。更准确地说,
    转换模型可以定义为一个函数。当接收状态和行动作为输入时,返回执行该行动后得到的\textbf{新状态}。例如,在15拼图游戏中,
    给定某个配置(状态),移动方块(行动)会导致拼图的新配置(新状态)。\smallskip
    \item \textbf{状态空间(State Space)}:\\从初始状态通过任何一系列行动可以到达的所有状态的\textbf{集合}。
    例如,在15拼图中,状态空间包含所有可以从任意初始状态通过一系列合法行动达到的$\frac{16!}{2} $种棋盘配置。
    状态空间可以被看作是一个有向图,\textbf{其中状态是节点,行动是节点之间的箭头。}\smallskip
    \begin{figure}[htbp]
        \centering
         \includegraphics[width=0.8\linewidth]{2.eps}
         \caption{状态空间}
         \label{zhuangtaikongjian}
      \end{figure}
\end{itemize}
\subsection{DFS与BFS}
\begin{itemize}
    \item \textbf{深度优先算法(DFS)}:\\
    深度优先搜索算法在尝试另一个方向之前会穷尽每个方向。在这些情况下 \textbf{frontier} 作为\textbf{堆栈数据结构}进行管理。
    需要记住的标语是\textbf{“后进先出”}。将节点添加到边界后,要删除并考虑的第一个节点是要添加的最后一个节点。\textbf{这导致搜索算法在第一个方向上尽可能深入,而将所有其他方向留给以后}。\\
一个例子:以你正在寻找钥匙的情况为例。在深度优先的搜索方法中,如果选择从裤子开始搜索,你首先会检查每个口袋,清空每个口袋并仔细检查内容。只有当你完全用尽了裤子每个口袋的搜索后,你才会停止在裤子里搜索,并开始在其他地方搜索。
\smallskip
    \begin{itemize}
    \item \textbf{优点:}
    此算法充其量是最快的。如果它 “运气好” 并且总是 (偶然) 选择正确的解决方案路径,
    那么深度优先搜索需要尽可能短的时间来找到解决方案。
    \item \textbf{缺点:}
    找到的解决方案可能不是最优的。
    在最坏的情况下,此算法将在找到解决方案之前探索所有可能的路径,从而在获得解决方案之前花费尽可能长的时间。
    \end{itemize}
    \item \textbf{深度优先算法(BFS)}:\\
    广度优先搜索算法将同时遵循多个方向,在每个可能的方向上迈出一步,然后在每个方向上迈出第二步。在这种情况下,frontier 作为\textbf{队列数据结构}进行管理。
    需要记住的标语是\textbf{“先进先出”}。在这种情况下,所有新节点都会按顺序加起来,并且根据先添加的节点来考虑节点
    \textbf{先到先得!这会导致搜索算法在每个可能的方向上采取一步,然后再在任何一个方向上采取第二步}。\\
一个例子:假设正在寻找钥匙。在这种情况下,如果你从裤子开始,你会看看你的右口袋。在此之后,你将查看一个抽屉,而不是查看您的左口袋。然后在桌子上。以此类推,在你能想到的每个位置。只有在你用完所有位置后,你才会回到你的裤子里,在下一个口袋里搜索。
\smallskip    
    \begin{itemize}
        \item \textbf{优点:}
        此算法保证找到最佳解决方案。
        \item \textbf{缺点:}
        几乎可以保证此算法的运行时间比最短时间长。
        在最坏的情况下,此算法需要尽可能长的运行时间。
    \end{itemize}

\end{itemize}
\subsection{\textbf{贪心算法和A*算法}}
广度优先和深度优先都是\textbf{不知情}的搜索算法。也就是说,
这些算法没有利用他们没有通过自己的探索获得的有关问题的任何知识。但是,
大多数情况下,实际上可以获得有关该问题的一些知识。例如,当人类迷宫求解器进入一个路口时,
人类可以看到哪条路沿着解决方案的大致方向走,哪条路不走。(总是选择尽量靠近出口的方向选择)
一种考虑额外知识以尝试提高其性能的算法称为\textbf{知情搜索算法}。
\begin{itemize}
    \item \textbf{贪心算法}: \\
    \textbf{贪婪最佳优先搜索}会扩展最接近目标的节点,由\textbf{启发式函数$h(n)$} 确定。顾名思义,
    该函数估计下一个节点离目标有多近,但可能会出错。贪婪最佳优先算法的效率取决于启发式函数的好坏。\\
    例如,在迷宫中,算法可以使用启发式函数,该函数依赖于可能节点与迷宫末端之间的\textbf{曼哈顿距离}。
    曼哈顿距离忽略墙壁,并计算从一个位置到目标位置需要向上、向下或向两侧走多少步。这是一个简单的估计值,
    可以根据当前位置和目标位置的 $(x,y) $坐标得出。

    \begin{figure}[htbp]
        \centering
         \includegraphics[width=0.8\linewidth]{1.eps}
         \caption{曼哈顿距离}
         \label{manhadun}
      \end{figure}

    \item \textbf{A*算法}: \\作为贪婪最佳优先算法的发展,$A*$ 搜索不仅考虑 $h(n)$(从当前位置到目标的估计成本),
    还考虑 $g(n)$(直到当前位置为止的\textbf{累积成本})。通过组合这两个值,该算法可以更准确地确定解决方案的成本并随时随地优化其选择。
    该算法会跟踪 (到目前为止的\textbf{路径成本 + 达到目标的估计成本}),一旦它超过前一个选项的估计成本,算法将放弃当前路径并返回上一个选项,
    从而防止自身走上一条 $h(n)$ 错误标记为最佳路径的漫长、低效的路径。\\
    再一次,由于此算法也依赖于启发式方法,因此它与它采用的启发式方法一样好。在某些情况下,它可能不如贪婪的最好优先搜索甚至不知情的算法有效。
    要使 A* 搜索最优,启发式函数 $h(n)$ 应为:
    \begin{itemize}
        \item \textbf{可接受},或从不高估真实成本,以及
        \item \textbf{一致},这意味着除了从前一个节点过渡到新节点的目标的成本外,到新节点目标的估计路径成本大于或等于到前一个节点目标的估计路径成本。
        用方程式来说,如果对于每个节点 n 和后继节点 n',
        步长成本为 c,$h(n) ≤ h(n') + c$,则 $h(n)$ 是一致的。
    \end{itemize}
    \end{itemize}


\section{\textbf{LU分解}}
LU分解是指矩阵$A$可以分解为 $LU$乘积的形式,其中$L$是单位下三角矩阵,$U$是单位上三角矩阵。\\
实际上就是用$LU$矩阵保存下来高斯消元的行列线性变换,利于求解$Ax=b$,在$b$不确定时将计算从$n^3$化简成$n^2$
\subsection{textbf{Guass变换}}
若想将给定矩阵 $A$
 分解为下三角矩阵 $L$
 和上三角矩阵 $U$
 ,一个思路就是通过一系列的初等变换将$A$ 
 化为上三角矩阵,且保证这些变换的乘积是一个下三角,比如通过初等变换 $L_{n-1} L_{n-2} ...L_{1}A=U$
 ,则 $A=L_{1}^{-1}L_{2}^{-1}...L_{n-1}^{-1}U$
,其中 U
 是一个上三角矩阵,  $(L_{1}L_{2}...L{n})^{-1}$
 是一个下三角矩阵。所以问题就转化为找满足条件的下三角矩阵,对于任意给定的向量 $x\in R^{n}$
 ,找一个简单的下三角矩阵 $L_{k}$
 使 $x$
 经过这一矩阵的作用之后的第k+1至第n个分量均为0。能够完成这一条件的最简单的下三角矩阵如下:
 \begin{equation}
    L^k = I-l_{k}e_{k}^{T}
 \end{equation}
 \begin{equation}
    l_{k}=\left(0 \ldots 0, l_{k+1, k} \ldots l_{n k}\right)^{T}
 \end{equation}
 \begin{equation}
   L_{k}=\left(\begin{array}{cccc}
        1 & \ddots & & \\
        & 1 & & \\
        & -l_{k+1, k} & 1 & \ddots \\
        & \vdots & & \\
        & -l_{n, k} & & 1
        \end{array}\right)
 \end{equation}
 \subsection{\text{高斯消元法实现LU分解}}
 对于任意给定的向量 \( x = (x_1, \ldots, x_n)^T \in R^n \),
 则  
 \[
 L_k x = (x_1, \ldots, x_k, x_{k+1} - x_k l_{k+1, k}, \ldots, x_n - x_k l_{n, k})
 \]  
 
 若要使得第 \( k+1 \) 至第 \( n \) 个分量均为 0,
 则  
 \[
 l_{ik} = \frac{x_i}{x_k}, \quad i = k+1, \ldots, n
 \]  
 
 此时  
 \[
 L_k x = (x_1, \ldots, x_k, 0, \ldots, 0)
 \]  
 
 且 Gauss 变换 \( L_k \) 的性质非常好,Gauss 变换的逆特别容易求:
 \[
 (I - l_k e_k^T)(I + l_k e_k^T) = I - l_k e_k^T l_k e_k^T = I
 \]  
 即  
 \[
 L_k^{-1} = I + l_k e_k^T
 \]  
 
 此时  
 \[
 A = L_1^{-1} L_2^{-1} \cdots L_n^{-1} U = LU
 \]  
 
 其中  
 \[
 L = L_1^{-1} L_2^{-1} \cdots L_n^{-1} = (I + l_1 e_1^T)(I + l_2 e_2^T) \cdots (I + l_{n-1} e_{n-1}^T)
 \]
 \[
 = I + l_1 e_1^T + l_2 e_2^T + \cdots + l_{n-1} e_{n-1}^T
 \]  
 
 则 \( L \) 具有如下形式:
 \[
 L = I + (l_1, l_2, \ldots, l_{n-1}, 0) = 
 \begin{pmatrix}
 1      &        &        &        & \\
 l_{21} & 1      &        &        & \\
 l_{31} & l_{32} & 1      &        & \\
 \vdots & \vdots & \vdots & \ddots & \\
 l_{n1} & l_{n2} & l_{n3} & \cdots & 1
 \end{pmatrix}
 \]  
 
 所以 \( L \) 是一个单位下三角矩阵,矩阵 \( A \) 可以分解为单位下三角矩阵 \( L \) 和上三角矩阵 \( U \) 的形式。
 
\subsection{\textbf{用法}}
知道了$L$和$U$,式子$Ax=b$可以写作$LUx=b$,写作$Ly=b$,先解出$y$这个过程称为\textbf{正向迭代}
然后再由$Ux=y$解出$x$,此过程称之为\textbf{反向迭代}。如此就不需要在每次在$b$发生变化时多次进行重复的高斯消元,简化计算机计算量



\section{总结}
\begin{itemize}
    \item [1]最短路径算法虽然从迷宫的角度来理解还是较为直观和轻松,但是在代码层面实现起来还是需要
    一定时间去理解。特别是当这类算法抽象用来解决一些表面上看起来不是很直观的问题上面。还得多多学习
    \item [2]关于矩阵的计算问题以后也是会经常用的,在搭建网络的时候很有用处。比如课上老师所展示的谷歌的
    网页排序的问题。就可以用概率矩阵和最初始的停留在网页链接上的人数推断出最终达到稳态时网页停留人数排名
    \item [3]这周因为临近的考试,还有学习latex的时间,团课啥啥的,导致学的东西除了课内的知识外并不是特别多
    还是得多多抽出时间学习课外内容,把效率提高。
\end{itemize}


% {
%     \small
%     \bibliographystyle{ieeenat_fullname}
%     \bibliography{main}
% }

% WARNING: do not forget to delete the supplementary pages from your submission 
% \clearpage
\setcounter{page}{1}
\maketitlesupplementary


\section{Rationale}
\label{sec:rationale}
% 
Having the supplementary compiled together with the main paper means that:
% 
\begin{itemize}
\item The supplementary can back-reference sections of the main paper, for example, we can refer to \cref{sec:intro};
\item The main paper can forward reference sub-sections within the supplementary explicitly (e.g. referring to a particular experiment); 
\item When submitted to arXiv, the supplementary will already included at the end of the paper.
\end{itemize}
% 
To split the supplementary pages from the main paper, you can use \href{https://support.apple.com/en-ca/guide/preview/prvw11793/mac#:~:text=Delete%20a%20page%20from%20a,or%20choose%20Edit%20%3E%20Delete).}{Preview (on macOS)}, \href{https://www.adobe.com/acrobat/how-to/delete-pages-from-pdf.html#:~:text=Choose%20%E2%80%9CTools%E2%80%9D%20%3E%20%E2%80%9COrganize,or%20pages%20from%20the%20file.}{Adobe Acrobat} (on all OSs), as well as \href{https://superuser.com/questions/517986/is-it-possible-to-delete-some-pages-of-a-pdf-document}{command line tools}.

\end{document}
